\documentclass[11pt]{article}

% NOTE: The "Edit" sections are changed for each assignment

% Edit these commands for each assignment

\newcommand{\assignmentduedate}{March 25}
\newcommand{\assignmentassignedate}{March 11}
\newcommand{\assignmentnumber}{Six}

\newcommand{\labyear}{2019}
\newcommand{\labday}{Monday}
\newcommand{\labtime}{2:30 pm}

\newcommand{\assigneddate}{Assigned: \labday, \assignmentassignedate, \labyear{} at \labtime{}}
\newcommand{\duedate}{Due: \labday, \assignmentduedate, \labyear{} at \labtime{}}

% Edit these commands to give the name to the main program

\newcommand{\mainprogram}{\lstinline{index.html}}
\newcommand{\mainprogramsource}{\lstinline{src/www/index.html}}

\newcommand{\secondprogram}{\lstinline{table.css}}
\newcommand{\secondprogramsource}{\lstinline{src/www/css/table.css}}

% Edit this commands to describe key deliverables

\newcommand{\reflection}{\lstinline{writing/reflection.md}}

% Commands to describe key development tasks

% --> Running gatorgrader.sh
\newcommand{\gatorgraderstart}{\command{gradle grade}}
\newcommand{\gatorgradercheck}{\command{gradle grade}}

% Commands to describe key git tasks

% NOTE: Could be improved, problems due to nesting

\newcommand{\gitcommitfile}[1]{\command{git commit #1}}
\newcommand{\gitaddfile}[1]{\command{git add #1}}

\newcommand{\gitadd}{\command{git add}}
\newcommand{\gitcommit}{\command{git commit}}
\newcommand{\gitpush}{\command{git push}}
\newcommand{\gitpull}{\command{git pull}}

\newcommand{\gitcommitmainprogram}{\command{git commit src/www/index.html -m "Your
descriptive commit message"}}

% Use this when displaying a new command

\newcommand{\command}[1]{``\lstinline{#1}''}
\newcommand{\program}[1]{\lstinline{#1}}
\newcommand{\url}[1]{\lstinline{#1}}
\newcommand{\channel}[1]{\lstinline{#1}}
\newcommand{\option}[1]{``{#1}''}
\newcommand{\step}[1]{``{#1}''}

\usepackage{pifont}
\newcommand{\checkmark}{\ding{51}}
\newcommand{\naughtmark}{\ding{55}}

\usepackage{listings}
\lstset{
  basicstyle=\small\ttfamily,
  columns=flexible,
  breaklines=true
}

\usepackage{fancyhdr}

\usepackage[margin=1in]{geometry}
\usepackage{fancyhdr}

\pagestyle{fancy}

\fancyhf{}
\rhead{Computer Science 302}
\lhead{Laboratory Assignment \assignmentnumber{}}
\rfoot{Page \thepage}
\lfoot{\duedate}

\usepackage{titlesec}
\titlespacing\section{0pt}{6pt plus 4pt minus 2pt}{4pt plus 2pt minus 2pt}

\newcommand{\labtitle}[1]
{
  \begin{center}
    \begin{center}
      \bf
      CMPSC 302\\Web Development\\
      Spring 2019\\
      \medskip
    \end{center}
    \bf
    #1
  \end{center}
}

\begin{document}

\thispagestyle{empty}

\labtitle{Laboratory \assignmentnumber{} \\ \assigneddate{} \\ \duedate{}}

\section*{Objectives}

To learn how to write HTML and CSS files that include headers, content, tags,
style rules, and a form, specifically implementing a ``travel survey'' web site
leveraging the example in Figure 3.9 of the textbook. As in previous
assignments, you will create your travel web site with features such as the use
of a \program{<table>} tag and emoji loaded from a CSS library, additionally
applying a style that adopts Google Web Fonts, customized font sizes, and
borders for a \program{<footer>}. As in a previous assignment, you will add an
HTML table that is styled by additional CSS rules. Specifically, your HTML table
will feature a bold font for the header, alternating shading for its rows, and
interactivity evident when the user hovers over a cell. As the specific focus of
this assignment, you will also use an HTML form and form control elements to
create a survey about international travel that will use Formspree.io to provide
a response by email. Finally, you will continue to practice running a web server
and using an automated tool to assess your completion of a project.

\section*{Suggestions for Success}

\begin{itemize}
  \setlength{\itemsep}{0pt}

\item {\bf Use the laboratory computers}. The computers in this laboratory feature specialized software for completing
  this course's laboratory and practical assignments. If it is necessary for you to work on a different machine, be sure
  to regularly transfer your work to a laboratory machine so that you can check its correctness. If you cannot use a
  laboratory computer and you need help with the configuration of your own laptop, then please carefully explain its
  setup to a teaching assistant or the course instructor when you are asking questions.

\item {\bf Follow each step carefully}. Slowly read each sentence in the assignment sheet, making sure that you
  precisely follow each instruction. Take notes about each step that you attempt, recording your questions and ideas and
  the challenges that you faced. If you are stuck, then please tell a teaching assistant or instructor what assignment
  step you recently completed.

\item {\bf Regularly ask and answer questions}. Please log into Slack at the start of a laboratory or practical session
  and then join the appropriate channel. If you have a question about one of the steps in an assignment, then you can
  post it to the designated channel. Or, you can ask a student sitting next to you or talk with a teaching assistant or
  the course instructor.

\item {\bf Store your files in GitHub}. Starting with this laboratory assignment, you will be responsible for storing
  all of your files (e.g., JavaScript code and Markdown-based writing) in a Git repository using GitHub Classroom.
  Please verify that you have saved your source code in your Git repository by using \command{git status} to ensure that
  everything is updated. You can see if your assignment submission meets the established correctness requirements by
  using the provided checking tools on your local computer and in checking the commits in GitHub.

\item {\bf Keep all of your files}. Don't delete your programs, output files, and written reports after you submit them
  through GitHub; you will need them again when you study for the quizzes and examinations and work on the other
  laboratory, practical, and final project assignments.

\item {\bf Back up your files regularly}. All of your files are regularly backed-up to the servers in the Department of
  Computer Science and, if you commit your files regularly, stored on GitHub. However, you may want to use a flash
  drive, Google Drive, or your favorite backup method to keep an extra copy of your files on reserve. In the event of
  any type of system failure, you are responsible for ensuring that you have access to a recent backup copy of all your
  files.

\item {\bf Explore teamwork and technologies}. While certain aspects of the laboratory assignments will be challenging
  for you, each part is designed to give you the opportunity to learn both fundamental concepts in the field of computer
  science and explore advanced technologies that are commonly employed at a wide variety of companies. To explore and
  develop new ideas, you should regularly communicate with your team and/or the teaching assistants and tutors.

\item {\bf Hone your technical writing skills}. Computer science assignments require to you write technical
  documentation and descriptions of your experiences when completing each task. Take extra care to ensure that your
  writing is interesting and both grammatically and technically correct, remembering that computer scientists must
  effectively communicate and collaborate with their team members and the tutors, teaching assistants, and course
  instructor.

% \item {\bf Review the Honor Code on the syllabus}. While you may discuss your
  % assignments with others, copying source code or writing is a violation of
  % Allegheny College's Honor Code.

\end{itemize}

\section*{Reading Assignment}

If you have not done so already, please read all of the relevant ``GitHub
Guides'', available at \url{https://guides.github.com/}, that explain how to use
many of the features that GitHub provides. In particular, make sure that you
have read guides such as ``Mastering Markdown'' and ``Documenting Your Projects
on GitHub''; each of them will help you to understand how to use GitHub. To do
well on this assignment, you should also review Chapters 1 and 5 in the course
textbook, paying particularly close attention to Sections 5.3 through 5.5 and
Figures 5.18, 5.19, and 5.20. Please see the course instructor or a teaching
assistant if you have questions about these reading assignments.

\section*{Accessing the Laboratory Assignment on GitHub}

To access the laboratory assignment, you should go into the
\channel{\#announcements} channel in our Slack team and find the announcement
that provides a link for it. Copy this link and paste it into your web browser.
Now, you should accept the laboratory assignment and see that GitHub Classroom
created a new GitHub repository for you to access the assignment's starting
materials and to store the completed version of your assignment. Specifically,
to access your new GitHub repository for this assignment, please click the green
``Accept'' button and then click the link that is prefaced with the label ``Your
assignment has been created here''. If you accepted the assignment and correctly
followed these steps, you should have created a repository with a name like
``Allegheny-Computer-Science-302-Spring-2019/computer-science-302-spring-2019-lab-6-gkapfham''.
Unless you provide the course instructor with documentation of the extenuating
circumstances that you are facing, not accepting the assignment means that you
automatically receive a failing grade for it.

Before you move to the next step of this assignment, please make sure that you
read all of the content on the web site for your new GitHub repository, paying
close attention to the technical details about the commands that you will type
and the content that must be evident in your web site. Now you are ready to
download the starting materials to your laboratory computer. Click the ``Clone
or download'' button and, after ensuring that you have selected ``Clone with
SSH'', please copy this command to your clipboard. To enter into your course
directory you should now type \command{cd cs302S2019}. Now, by typing
\command{git clone} in your terminal and then pasting in the string that you
copied from the GitHub site you will download all of the code for this
assignment. For instance, if the course instructor ran the \command{git clone}
command in the terminal, it would look like:

\begin{lstlisting}
  git clone git@github.com:Allegheny-Computer-Science-302-S2019/computer-science-302-spring-2019-lab-6-gkapfham.git
\end{lstlisting}

After this command finishes, you can use \command{cd} to change into the new
directory. If you want to \step{go back} one directory from your current
location, then you can type the command \command{cd ..}. Please continue to use
the \command{cd} and \command{ls} commands to explore the files that you
automatically downloaded from GitHub. What files and directories do you see?
What do you think is their purpose? Please note that this assignment includes
both an HTML and a CSS file with starting content. Spend some time exploring
these files, sharing your discoveries with a \mbox{teaching assistant}.

\section*{Creating an Interactive Survey with an HTML Form}

This laboratory assignment invites you to implement a web site using the HTML
and CSS programming languages, building on the work that you completed during
previous assignments. In particular, you will create an extended version of the
travel photographs web site that partially looks like the one in Figure 3.9. Now
you will include an HTML form that administers a survey about international
travel. Please refer to your GitHub repository for a screenshot that shows what
your final site should look like when completed. Additionally, please note that
this assignment still asks you to complete a site that contains an HTML table
that you style with CSS. Overall, you will need to read and resolve all of the
\command{TODO} markers in the provided HTML and CSS files. To start, you should
include the cascading style sheet (CSS) that controls, for instance, the font
sizes on the site. You can achieve this step by adding the code \program{<link
href="css/site.css" rel="stylesheet">} to the \program{<head>} region of the
\mainprogramsource{} file. Note that you also need to have another line of
source code to load the \program{emoji.css} and \program{table.css} files and
their features for further styling the web site. How would you write these lines
in the \program{<head>} region of the HTML source code?

\begin{figure}[t]
  \centering
  \begin{verbatim}
    <!-- Follow instructions at https://formspree.io/ to start your form -->
    <form action="https://formspree.io/YOUREMAIL@allegheny.edu" method="POST">

      <!-- Specify the server location to avoid page reload problems -->
      <input type="hidden" name="_next" value="http://localhost:4250/"/>

      <!-- Create fields for your name and email -->
      <label>Name</label> <input type="text" name="name"> <br>
      <label>Email</label> <input type="email" name="_replyto"> <br> <br>

      <!-- Follow the instructions to create the rest of the form -->
    </form>
  \end{verbatim}
  \vspace*{-.35in}
  \caption{The HTML Source Code to Create the Form that Interacts with Formspee.}~\label{fig:form}
  \vspace*{-.25in}
\end{figure}

As you can see from the screenshot provided in your GitHub repository, your site
should have content like you added in a previous assignment. You can include
source code from another GitHub repository to ensure that your web site has all
of these features. For instance, you will need to use and style the
\program{<footer>} tag in your web site's source code. Remember, your footer
should have a border on the bottom and right, as seen in the screenshot. Also,
you should add the \program{<table>} tag and all of the header, row, and cell
markup tags so that you can produce a table that looks like the one in the
screenshot. To learn more about these tags you can review Section 5.1's content
in the textbook. Please talk to the instructor or a teaching assistant if you
have any questions. Now, you are ready to add in the new features that are
unique to this assignment. Specifically, you need to learn how to use the form
reporting tool available from \url{https://formspree.io/}.

Figure~\ref{fig:form} provide some samples of the HTML source code that you will
need to include in your form. For instance, the first line of code declares the
form and connects it to the Formspree service. After you authorize the use of
this service, it will enable you to receive an email each time someone submits
your form. The second line of source code ensures that the Formspree system will
``redirect'' back to your web site after the form is submitted to Formspree.
Additionally, you will see that Figure~\ref{fig:form} provides you with the
source code to display both the labels for the person's name and email address
and then the affiliated input fields. You can learn about all of the other
required form control elements by studying the \command{TODO} markers and
reading the textbook's description of Figures 5.18., 5.19, and 5.21. Make sure
that the form submits correctly and that it includes \program{<br>} tags to
enable proper formatting. Please see the instructor with any questions about
these tasks.

% Finally, students should remember to check the correctness of their HTML by
% running the command \command{htmlhint src/www/index.html} in their terminal
% window. If the web site displays incorrectly, please discuss the problems with a
% teaching assistant or the course instructor.

Now, you can run your web server by typing the command \command{serve src/www
4250}. At this point, you can start your web browser and go to the site
\url{http://localhost:4250/}. Please check to see if your new web site looks
correct. If it is not, then continue to edit and check it until the files are
correct. You should also ensure that, when you run the web server, it produces
output suggesting that it is returning three CSS files and an image. If not,
then please make sure that you revise your HTML source code and/or check the
configuration of your web server. Specifically, please make sure that your web
server is returning the \program{site.css} and \program{table.css} files that
contain the styles for your HTML. Finally, when you are interacting with your
web site you should verify that certain links open in a new browser tab, the
table's cells change color when you hover over them, and the form contains all
of the required components for the interactive survey that reports by email.

% Please see the instructor if you have questions about the CSS or HTML code.

As you complete this assignment, you should regularly commit files to your
GitHub repository, using the ``Git Cheat Sheet'' and following the steps that
are described in the next section. Also, note that your web server will require
a dedicated terminal when it is running. After completing the assignment,
reflect on the entire process. What step did you find to be the most
challenging? You should write your reflections in a file, called \reflection{},
that also uses Markdown. To complete this part of the assignment, you should
write one high-quality paragraph that reports on your experiences. Now, verbally
share your experiences with another class member and the instructor and at least
one the teaching assistants! Finally, please take the time to answer the other
questions in the \reflection{} file. For instance, make sure that you understand
and can explain how the HTML and CSS files work together to format the text and
enable the display of emoji footer at the bottom of the site. You should also be
able to explain how you used \program{<br>} and \program{<label>} tags to
correctly format and display the survey content in the web site's form.

\section*{Checking the Correctness of Your Web Site and Writing}

The Markdown file that contains your reflection must adhere to the standards
described in the Markdown Syntax Guide
\url{https://guides.github.com/features/mastering-markdown/}. Finally, your
\reflection{} file should adhere to the Markdown standards established by the
\step{Markdown linting} tool available at
\url{https://github.com/markdownlint/markdownlint/} and the writing standards
set by the \step{prose linting} tool from \url{http://proselint.com/}. Instead
of requiring you to manually check that your deliverables adhere to these
industry-accepted standards, the GatorGrader tool that you will use in this
laboratory assignment makes it easy for you to automatically check if your
submission meets the standards for correctness. For instance, GatorGrader will
check to ensure that \mainprogram{} has the required sections and an
interactive survey.

To get started with the use of GatorGrader, type the command \gatorgraderstart{}
in your terminal window. If you have mistakes in your assignment, then you will
need to review GatorGrader's output, find the mistake, and try to fix it. Once
your web page is displaying correctly, fulfilling at least some of the
assignment's requirements, you should transfer your files to GitHub using the
\gitcommit{} and \gitpush{} commands. For example, if you want to signal that
the \mainprogramsource{} file has been changed and is ready for transfer to
GitHub you would first type \gitcommitmainprogram{} in your terminal, followed
by typing \gitpush{} to perform the transfer.
%
If you notice that transferring your files to GitHub did not work correctly,
then please read the terminal messages and try to determine why.
%
Please see a teaching assistant or the course instructor if you have questions
about any of these steps!

After the course instructor enables \step{continuous integration} with a system
called Travis CI, when you use the \gitpush{} command to transfer your source
code to your GitHub repository, Travis CI will initialize a \step{build} of your
assignment, checking to see if it meets all of the requirements. If both your
source code and writing meet all of the established requirements, then you will
see a green \checkmark{} in the listing of commits in GitHub after awhile. If
your submission does not meet the requirements, a red \naughtmark{} will appear
instead. The instructor will reduce a student's grade for this assignment if the
red \naughtmark{} appears on the last commit in GitHub immediately before the
assignment's due date. Yet, if the green \checkmark{} appears on the last commit
in your GitHub repository, then you satisfied all of the main checks, thereby
allowing the course instructor to evaluate other aspects of your source code and
writing, as further described in the \step{Evaluation} section of this
assignment sheet. Unless you provide the instructor with documentation of the
extenuating circumstances that you are facing, no late work will be considered
towards your grade for this laboratory assignment.

\section*{Summary of the Required Deliverables}

\noindent Students do not need to submit printed source code or technical
writing for any assignment in this course. Instead, this assignment invites you
to submit, using GitHub, the following deliverables. Overall, your submitted web
site should match the screenshot provided in your GitHub repository.

\begin{enumerate}

  \setlength{\itemsep}{0in}

\item Stored in \reflection{}, a one-paragraph answer to all of the stated
  questions. For every challenge that you encountered, please explain your
  solution for it.

\item A properly formatted and correct version of \mainprogramsource{} that both
  meets all of the established requirements and contains the correct HTML and
  the desired static web site.

\item A properly formatted and correct version of \secondprogramsource{} that
  both meets all of the established requirements and contains the correct CSS
  for styling the table.

\end{enumerate}

\section*{Evaluation of Your Laboratory Assignment}

Using a report that the instructor shares with you through the commit log in
GitHub, you will privately received a grade on this assignment and feedback on
your submitted deliverables. Your grade for the assignment will be a function of
the whether or not it was submitted in a timely fashion and if your program
received a green \checkmark{} indicating that it met all of the requirements.
Other factors will also influence your final grade on the assignment. In
addition to studying the efficiency and effectiveness of your Markdown, the
instructor will also evaluate the accuracy of both your technical writing and
the contents of your source code. If your submission receives a red
\naughtmark{}, the instructor will reduce your grade for the assignment while
still considering the regularity with which you committed to your GitHub
repository and the overall quality of your partially completed work. Please see
the instructor if you have questions about the evaluation of this laboratory
assignment.

\section*{Adhering to the Honor Code}

In adherence to the Honor Code, students should complete this assignment on an
individual basis. While it is appropriate for students in this class to have
high-level conversations about the assignment, it is necessary to distinguish
carefully between the student who discusses the principles underlying a problem
with others and the student who produces assignments that are identical to, or
merely variations on, someone else's work. Deliverables (e.g., HTML source code
or Markdown-based technical writing) that are nearly identical to the work of
others will be taken as evidence of violating the \mbox{Honor Code}. Please see
the course instructor if you have questions about this policy.

\end{document}
