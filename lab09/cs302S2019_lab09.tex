\documentclass[11pt]{article}

% NOTE: The "Edit" sections are changed for each assignment

% Edit these commands for each assignment

\newcommand{\assignmentduedate}{April 22}
\newcommand{\assignmentassignedate}{April 15}
\newcommand{\assignmentnumber}{Nine}

\newcommand{\labyear}{2019}
\newcommand{\labday}{Friday}
\newcommand{\labdueday}{Monday}
\newcommand{\labtime}{9:00 am}

\newcommand{\assigneddate}{Assigned: \labday, \assignmentassignedate, \labyear{} at \labtime{}}
\newcommand{\duedate}{Due: \labdueday, \assignmentduedate, \labyear{} at \labtime{}}

% Edit these commands to give the name to the main program

\newcommand{\reflection}{\lstinline{writing/reflection.md}}

\newcommand{\mainprogram}{\lstinline{index.html}}
\newcommand{\mainprogramsource}{\lstinline{src/www/index.html}}

\newcommand{\secondprogram}{\lstinline{site.css}}
\newcommand{\secondprogramsource}{\lstinline{src/www/css/site.css}}

\newcommand{\thirdprogram}{\lstinline{country.js}}
\newcommand{\thirdprogramsource}{\lstinline{src/www/js/country.js}}

\newcommand{\screenshot}{\lstinline{screenshot.png}}
\newcommand{\screenshotsource}{\lstinline{images/javascript\_submission.png}}

% Commands to describe key development tasks

% --> Running gatorgrader.sh
\newcommand{\gatorgraderstart}{\command{gradle grade}}
\newcommand{\gatorgradercheck}{\command{gradle grade}}

% Commands to describe key git tasks

% NOTE: Could be improved, problems due to nesting

\newcommand{\gitcommitfile}[1]{\command{git commit #1}}
\newcommand{\gitaddfile}[1]{\command{git add #1}}

\newcommand{\gitadd}{\command{git add}}
\newcommand{\gitcommit}{\command{git commit}}
\newcommand{\gitpush}{\command{git push}}
\newcommand{\gitpull}{\command{git pull}}

\newcommand{\gitcommitmainprogram}{\command{git commit src/www/index.html -m "Your
descriptive commit message"}}

% Use this when displaying a new command

\newcommand{\command}[1]{``\lstinline{#1}''}
\newcommand{\program}[1]{\lstinline{#1}}
\newcommand{\url}[1]{\lstinline{#1}}
\newcommand{\channel}[1]{\lstinline{#1}}
\newcommand{\option}[1]{``{#1}''}
\newcommand{\step}[1]{``{#1}''}

\usepackage{pifont}
\newcommand{\checkmark}{\ding{51}}
\newcommand{\naughtmark}{\ding{55}}

\usepackage{listings}
\lstset{
  basicstyle=\small\ttfamily,
  columns=flexible,
  breaklines=true
}

\usepackage{fancyhdr}

\usepackage[margin=1in]{geometry}
\usepackage{fancyhdr}

\pagestyle{fancy}

\fancyhf{}
\rhead{Computer Science 302}
\lhead{Laboratory Assignment \assignmentnumber{}}
\rfoot{Page \thepage}
\lfoot{\duedate}

\usepackage{titlesec}
\titlespacing\section{0pt}{6pt plus 4pt minus 2pt}{4pt plus 2pt minus 2pt}

\newcommand{\labtitle}[1]
{
  \begin{center}
    \begin{center}
      \bf
      CMPSC 302\\Web Development\\
      Spring 2019\\
      \medskip
    \end{center}
    \bf
    #1
  \end{center}
}

\begin{document}

\thispagestyle{empty}

\labtitle{Laboratory \assignmentnumber{} \\ \assigneddate{} \\ \duedate{}}

\section*{Objectives}

As a step towards learning more about JavaScript and how it relates to HTML and
CSS, you will create your own version of the ``Extended Example'' that starts on
page 383 of the textbook. Specifically, your web site will feature five
rectangular boxes, each with dark grey borders, that have the flag of a country
and then the required details about that country. Unlike in other assignments,
you will use JavaScript to automatically generate the HTML source code that
links to images.

\section*{Suggestions for Success}

\begin{itemize}
  \setlength{\itemsep}{0pt}

\item {\bf Use the laboratory computers}. The computers in this laboratory feature specialized software for completing
  this course's laboratory and practical assignments. If it is necessary for you to work on a different machine, be sure
  to regularly transfer your work to a laboratory machine so that you can check its correctness. If you cannot use a
  laboratory computer and you need help with the configuration of your own laptop, then please carefully explain its
  setup to a teaching assistant or the course instructor when you are asking questions.

\item {\bf Follow each step carefully}. Slowly read each sentence in the assignment sheet, making sure that you
  precisely follow each instruction. Take notes about each step that you attempt, recording your questions and ideas and
  the challenges that you faced. If you are stuck, then please tell a teaching assistant or instructor what assignment
  step you recently completed.

\item {\bf Regularly ask and answer questions}. Please log into Slack at the start of a laboratory or practical session
  and then join the appropriate channel. If you have a question about one of the steps in an assignment, then you can
  post it to the designated channel. Or, you can ask a student sitting next to you or talk with a teaching assistant or
  the course instructor.

\item {\bf Store your files in GitHub}. Starting with this laboratory
  assignment, you will be responsible for storing all of your files (e.g., HTML
  or CSS code and Markdown-based writing) in a Git repository using GitHub
  Classroom. Please verify that you have saved your source code in your Git
  repository by using \command{git status} to ensure that everything is updated.
  You can see if your assignment submission meets the established correctness
  requirements by using the provided checking tools on your local computer and
  in checking the commits in GitHub.

\item {\bf Keep all of your files}. Don't delete your programs, output files, and written reports after you submit them
  through GitHub; you will need them again when you study for the quizzes and examinations and work on the other
  laboratory, practical, and final project assignments.

\item {\bf Back up your files regularly}. All of your files are regularly backed-up to the servers in the Department of
  Computer Science and, if you commit your files regularly, stored on GitHub. However, you may want to use a flash
  drive, Google Drive, or your favorite backup method to keep an extra copy of your files on reserve. In the event of
  any type of system failure, you are responsible for ensuring that you have access to a recent backup copy of all your
  files.

\item {\bf Explore teamwork and technologies}. While certain aspects of the laboratory assignments will be challenging
  for you, each part is designed to give you the opportunity to learn both fundamental concepts in the field of computer
  science and explore advanced technologies that are commonly employed at a wide variety of companies. To explore and
  develop new ideas, you should regularly communicate with your team and/or the teaching assistants and tutors.

\item {\bf Hone your technical writing skills}. Computer science assignments require to you write technical
  documentation and descriptions of your experiences when completing each task. Take extra care to ensure that your
  writing is interesting and both grammatically and technically correct, remembering that computer scientists must
  effectively communicate and collaborate with their team members and the tutors, teaching assistants, and course
  instructor.

% \item {\bf Review the Honor Code on the syllabus}. While you may discuss your
%   assignments with others, copying source code or writing is a violation of
%   Allegheny College's Honor Code.

\end{itemize}

\section*{Reading Assignment}

If you have not done so already, please read all of the relevant ``GitHub
Guides'', available at \url{https://guides.github.com/}, that explain how to use
many of the features that GitHub provides. In particular, make sure that you
have read guides such as ``Mastering Markdown'' and ``Documenting Your Projects
on GitHub''; each of them will help you to understand how to use GitHub. To do
well on this assignment, you should also review Chapters 1 through 8 in the
course textbook, paying close attention to the JavaScript content in Sections
8.1 through 8.10 and Chapter 8's ``Extended Example''. Students who want to
learn more about JavaScript are also strongly encourage to review Chapters 9 and
20. Importantly, students are responsible for finding and reading all of the
auxiliary materials that they need to successfully design the requested web
site. Please see the course instructor or a teaching assistant if you have
questions about these reading assignments.

\section*{Accessing the Laboratory Assignment on GitHub}

To access the laboratory assignment, you should go into the
\channel{\#announcements} channel in our Slack team and find the announcement
that provides a link for it. Copy this link and paste it into your web browser.
Now, you should accept the laboratory assignment and see that GitHub Classroom
created a new GitHub repository for you to access the assignment's starting
materials and to store the completed version of your assignment. Specifically,
to access your new GitHub repository for this assignment, please click the green
``Accept'' button and then click the link that is prefaced with the label ``Your
assignment has been created here''. If you accepted the assignment and correctly
followed these steps, you should have created a repository with a name like
``Allegheny-Computer-Science-302-Spring-2019/computer-science-302-spring-2018-lab-9-gkapfham''.
Unless you provide the instructor with documentation of the extenuating
circumstances that you are facing, not accepting the assignment means that you
automatically receive a failing grade for it.

Before you move to the next step of this assignment, please make sure that you
read all of the content on the web site for your new GitHub repository, paying
close attention to the technical details about the commands that you will type
and the content that must be evident in your web site. Now you are ready to
download the starting materials to your laboratory computer. Click the ``Clone
or download'' button and, after ensuring that you have selected ``Clone with
SSH'', please copy this command to your clipboard. To enter into your course
directory you should now type \command{cd cs302S2019}. Now, by typing
\command{git clone} in your terminal and then pasting in the string that you
copied from the GitHub site you will download all of the code for this
assignment. For instance, if the course instructor ran the \command{git clone}
command in the terminal, it would look like:

\begin{lstlisting}
  git clone git@github.com:Allegheny-Computer-Science-302-S2019/computer-science-302-spring-2018-lab-9-gkapfham.git
\end{lstlisting}

After this command finishes, you can use \command{cd} to change into the new
directory. If you want to \step{go back} one directory from your current
location, then you can type the command \command{cd ..}. Please continue to use
the \command{cd} and \command{ls} commands to explore the files that you
automatically downloaded from GitHub. What files and directories do you see? In
what ways will you have to extend the JavaScript and HTML and add CSS?
Ultimately, you should aim to make your web site look like a completed version
of the one that is displayed on page 383 of the textbook.

\section*{Creating a Web Site that Displays Country Information}

As you start editing the source code, please review the content that is on pages
383 and 384 of the textbook. Then, you should study the \command{TODO} markers
in the provided source code. In the end, you will create a web site that looks
like the one on page 383. However, you will use the flag images that are
provided by the GitHub repository mentioned in the \thirdprogram{} file. Also,
you will discover and include the full details about five additional countries
that are not mentioned in the textbook. It is important to note that the
textbook's label written as ``{\bf iso}'' stands for the ISO-3166 country code
that was assigned to each country by the International Standards Organization.
You can use this two-letter code to find the required flag image for each
country. In summary, you must modify the HTML so that it links to the
JavaScript. Then, you will enhance the JavaScript so that it produces output for
five distinct countries. Finally, you will create a CSS file that styles the
country boxes as they are given on page 383. Please study the provided source
code and/or talk with the course instructor if you have questions about the
steps for completing this assignment.

% As you complete this practical assignment, you should regularly commit files to
% your GitHub repository, using the ``Git Cheat Sheet'' and the steps that you
% used in previous assignments and that are reviewed in the next section. Also,
% note that your web server will require a dedicated terminal when it is running.

% Students should also remember to check the correctness of their HTML by running
% the command \command{htmlhint src/www/index.html} in their terminal window.

Remember, you can run your web server by typing the command \command{serve
src/www 4250}. At this point, you can start your browser and go to the site
\url{http://localhost:4250/}. Please check to see if your web site looks
correct. If it is not, then continue to edit and check it until the files are
correct. You should also ensure that, when you run the web server, it produces
output suggesting that it is returning the correct files and images. If not,
then please make sure that you revise your HTML and/or JavaScript source code
and/or check the configuration of your web server. Finally, when you are
studying your web site you should verify the correctness of all its components
(e.g., your web site should feature appropriately styled details about five
countries in total, including the flag, ISO-3166 country code, current capital,
and an estimate of the current population).

\section*{Checking the Correctness of Your Web Site and Technical Writing}

The Markdown file that contains your reflection must adhere to the standards
described in the Markdown Syntax Guide
\url{https://guides.github.com/features/mastering-markdown/}. Finally, your
\reflection{} file should adhere to the Markdown standards established by the
\step{Markdown linting} tool available at
\url{https://github.com/markdownlint/markdownlint/} and the writing standards
set by the \step{prose linting} tool from \url{http://proselint.com/}. Instead
of requiring you to manually check that your deliverables adhere to these
industry-accepted standards, the GatorGrader tool that you will use in this
laboratory assignment makes it easy for you to automatically check if your
submission meets the standards for correctness. For instance, GatorGrader will
check to ensure that certain files and directories exist in your repository.
Since this assignment asks you to independently create your source code,
GatorGrader does not check for specific code.

To get started with the use of GatorGrader, type the command \gatorgraderstart{}
in your terminal window. If you have mistakes in your assignment, then you will
need to review GatorGrader's output, find the mistake, and try to fix it. Once
your web page is displaying correctly, fulfilling at least some of the
assignment's requirements, you should transfer your files to GitHub using the
\gitcommit{} and \gitpush{} commands. For example, if you want to signal that
the \mainprogramsource{} file has been changed and is ready for transfer to
GitHub you would first type \gitcommitmainprogram{} in your terminal, followed
by typing \gitpush{} to perform the transfer.
%
If you notice that transferring your files to GitHub did not work correctly,
then please read the terminal messages and try to determine why.
%
Please see a teaching assistant or the course instructor if you have questions
about any of these steps!

After the course instructor enables \step{continuous integration} with a system
called Travis CI, when you use the \gitpush{} command to transfer your source
code to your GitHub repository, Travis CI will initialize a \step{build} of your
assignment, checking to see if it meets all of the requirements. If both your
source code and writing meet all of the established requirements, then you will
see a green \checkmark{} in the listing of commits in GitHub after awhile. If
your submission does not meet the requirements, a red \naughtmark{} will appear
instead. The instructor will reduce a student's grade for this assignment if the
red \naughtmark{} appears on the last commit in GitHub immediately before the
assignment's due date. Yet, if the green \checkmark{} appears on the last commit
in your GitHub repository, then you satisfied all of the main checks, thereby
allowing the course instructor to evaluate other aspects of your source code and
writing, as further described in the \step{Evaluation} section of this
assignment sheet. Unless you provide the instructor with documentation of the
extenuating circumstances that you are facing, no late work will be considered
towards your grade for this laboratory assignment.

\section*{Summary of the Required Deliverables}

\noindent To ensure that you can successfully create your own web sites that
feature JavaScript, HTML, and CSS, this laboratory provides minimal HTML and
JavaScript source code and invites you to add many of the required features
(e.g., {\tt Country} objects and the flag images).

% Please see the instructor if you have questions
% about this assignment's goals.

\vspace*{-.05in}

\begin{enumerate}

  \setlength{\itemsep}{0in}

\item A properly formatted and correct version of \mainprogramsource{} that both
  meets all of the established requirements and contains the correct HTML for
  the desired static web site.

\item A properly formatted and correct version of \secondprogramsource{} that
  both meets all of the established requirements and contains the correct CSS
  code to properly style country boxes.

\item A properly formatted and correct version of \thirdprogramsource{} that
  both meets all of the established requirements and contains the correct
  JavaScript code to create the HTML.

\item Stored in \screenshotsource{}, an image of the final version of your web
  site. Note that you can use Firefox's ``print web page'' feature to produce
  this screenshot.

\item Stored in \reflection{}, a one-paragraph answer to all of the stated
  questions.

\end{enumerate}

\vspace*{-.1in}

\section*{Evaluation of Your Laboratory Assignment}

Using a report that the instructor shares with you through the commit log in
GitHub, you will privately received a grade on this assignment and feedback on
your submitted deliverables. Your grade for the assignment will be a function of
the whether or not it was submitted in a timely fashion and if your program
received a green \checkmark{} indicating that it met all of the requirements.
Other factors will also influence your final grade on the assignment. In
addition to studying the efficiency and effectiveness of your Markdown, the
instructor will also evaluate the accuracy of both your technical writing and
the contents of your source code. If your submission receives a red
\naughtmark{}, the instructor will reduce your grade for the assignment while
still considering the regularity with which you committed to your GitHub
repository and the overall quality of your partially completed work. Please see
the instructor if you have questions about the evaluation of this laboratory
assignment.

\section*{Adhering to the Honor Code}

% While it is appropriate for students in this class to have high-level
% conversations about the assignment, it is necessary to distinguish carefully
% between the student who discusses the principles underlying a problem with
% others and the student who produces assignments that are identical to, or merely
% variations on, someone else's work.

In adherence to the Honor Code, students should complete this assignment on an
individual basis.
%
While acknowledging you are working from provided code, deliverables (e.g., HTML
source code or technical writing) that are nearly identical to the work of
others will be taken as evidence of violating the \mbox{Honor Code}. Please see
the instructor with any questions about this policy.

\end{document}
