\documentclass[11pt]{article}

% NOTE: The "Edit" sections are changed for each assignment

% Edit these commands for each assignment

\newcommand{\assignmentduedate}{May 3}
\newcommand{\assignmentassignedate}{February 22}
\newcommand{\assignmentnumber}{Ten}

\newcommand{\labyear}{2019}
\newcommand{\labday}{Friday}
\newcommand{\labdueday}{Friday}
\newcommand{\labtime}{9:00 am}
\newcommand{\labduetime}{11:59 pm}

\newcommand{\assigneddate}{Assigned: \labday, \assignmentassignedate, \labyear{} at \labtime{}}
\newcommand{\duedate}{Due: \labdueday, \assignmentduedate, \labyear{} at \labduetime{}}

% Edit this commands to describe key deliverables

\newcommand{\reflection}{\lstinline{writing/reflection.md}}
\newcommand{\timefile}{\lstinline{writing/time.md}}
\newcommand{\colorfile}{\lstinline{writing/color.md}}
\newcommand{\readme}{\lstinline{README.md}}

% Commands to describe key development tasks

% --> Running gatorgrader.sh
\newcommand{\gatorgraderstart}{\command{./gatorgrader.sh --start}}
\newcommand{\gatorgradercheck}{\command{./gatorgrader.sh --check}}

% --> Compiling and running program with gradle
\newcommand{\gradlebuild}{\command{gradle build}}
\newcommand{\gradlerun}{\command{gradle run}}

% Commands to describe key git tasks

\newcommand{\gitcommitfile}[1]{\command{git commit #1}}
\newcommand{\gitaddfile}[1]{\command{git add #1}}

\newcommand{\gitadd}{\command{git add}}
\newcommand{\gitcommit}{\command{git commit}}
\newcommand{\gitpush}{\command{git push}}
\newcommand{\gitpull}{\command{git pull}}

% Use this when displaying a new command

\newcommand{\command}[1]{``\lstinline{#1}''}
\newcommand{\program}[1]{\lstinline{#1}}
\newcommand{\url}[1]{\lstinline{#1}}
\newcommand{\channel}[1]{\lstinline{#1}}
\newcommand{\option}[1]{``{#1}''}
\newcommand{\step}[1]{``{#1}''}

\usepackage{pifont}
\newcommand{\checkmark}{\ding{51}}
\newcommand{\naughtmark}{\ding{55}}

\usepackage{listings}
\lstset{
  basicstyle=\small\ttfamily,
  columns=flexible,
  breaklines=true
}

\usepackage{fancyhdr}
\usepackage{fancyvrb}

\usepackage[margin=1in]{geometry}
\usepackage{fancyhdr}

\pagestyle{fancy}

\fancyhf{}
\rhead{Computer Science 302}
\lhead{Final Project}
\rfoot{Page \thepage}
\lfoot{\duedate}

\usepackage{titlesec}
\titlespacing\section{0pt}{6pt plus 4pt minus 2pt}{4pt plus 2pt minus 2pt}

\newcommand{\labtitle}[1]
{
  \begin{center}
    \begin{center}
      \bf
      CMPSC 302\\Web Development\\
      Spring 2019\\
      \medskip
    \end{center}
    \bf
    #1
  \end{center}
}

\begin{document}

\thispagestyle{empty}

\labtitle{Web Design Portfolio \\ \assigneddate{} \\ \duedate{}}

\section*{Introduction}

Throughout the semester, you will explore the fundamentals of web development,
learning how to design and create mobile-ready web sites that contain, for
instance, images and forms and use the Markdown, HTML, CSS, and JavaScript
programming languages. This final project invites you to explore, in greater
detail, many of the  challenges and excitement of real-world web development.
%
Specifically, you will design, implement, test, and evaluate a mobile-ready site
that, whenever possible, engages with international and/or intercultural
perspectives and technologies. The goal of this project is for you to learn more
about how to use, implement, test, and evaluate different types of software for
creating real-world web sites.
%
Since you will implement and deploy your web site using GitHub and Netlify, you
will also learn more about version control, specifically focusing on the use of
the ``GitHub Flow'' model and the creation of branches, pull requests, and
Netlify preview builds.
%
You will create a fully documented web site that explores a stated theme and is
both publicly visible and mobile-ready.
%
Students are also invited to share the details about web design, implementation,
and deployment with their friends and instructors at Allegheny College.

In addition to featuring a production quality web site, your web design
portfolio will include a detailed plan for your project, three intermediate
status updates, and a final report.
%
Written in Markdown, the detailed plan should explain the theme for your web
site and then the steps that you will take to complete the final version of the
site.
%
Each of the three status reports will highlight your recent accomplishments and
your next steps in light of your detailed plan.
%
The final report should explain all of your source code, highlighting the key
contributions of your work. This report should also include a description of why
the chosen theme is important and discuss the design, implementation, and
testing that you undertook.
%
All of the written material in your web design portfolio should be appropriately
formatted and both grammatically and technically correct. The source code that
you write must be carefully documented and tested. If you use existing resources
(e.g., a Flickr image or a responsive web design framework), then the steps for
installation and use should be clearly documented in your report. Also, the
report must explain both the steps that you took to deploy your site to Netlify
and to run your own local web server and view the web site.

\section*{Suggestions for Success}

\begin{itemize}
  \setlength{\itemsep}{0pt}

\item {\bf Use the laboratory computers}. The computers in this laboratory
  feature specialized software for completing this course's laboratory and
  practical assignments. If it is necessary for you to work on a different
  machine, be sure to regularly transfer your work to a laboratory machine so
  that you can check its correctness. If you cannot use a laboratory computer
  and you need help with the configuration of your own laptop, then please
  carefully explain its setup to a teaching assistant or the course instructor
  when you are asking questions.

\item {\bf Store your files in GitHub}. As in the past laboratory assignments,
  you will be responsible for storing all of your files (e.g., HTML source code
  and Markdown-based writing) in a Git repository using GitHub Classroom. Please
  verify that you have saved your source code in your Git repository by using
  \command{git status} to ensure that everything is updated. You can see if your
  assignment submission meets the established correctness requirements by using
  the provided checking tools on your local computer and by checking the commits
  in GitHub.

\item {\bf Back up your files regularly}. All of your files are regularly
  backed-up to the servers in the Department of Computer Science and, if you
  commit your files regularly, stored on GitHub. However, you may want to use a
  flash drive, Google Drive, or your favorite backup method to keep an extra
  copy of your files on reserve. In the event of any type of system failure, you
  are responsible for ensuring that you have access to a recent backup copy of
  all your files.

\item {\bf Explore teamwork and technologies}. While certain aspects of the
  final project will be challenging for you, each part is designed to give you
  the opportunity to learn both fundamental concepts in the field of computer
  science and explore advanced technologies that are commonly employed at a wide
  variety of companies. To explore and develop new ideas, you should regularly
  communicate with your peers and/or the teaching assistants and tutors.

\item {\bf Hone your technical writing skills}. Computer science assignments
  require to you write technical documentation and descriptions of your
  experiences when completing each task. Take extra care to ensure that your
  writing is interesting and both grammatically and technically correct,
  remembering that computer scientists must effectively communicate and
  collaborate with their peers and the tutors, teaching assistants, and the
  course instructor.

\item {\bf Review all of your past laboratory and practical assignments}. Now
  that you have completed many prior assignments, please review all of your
  prior work to ensure that you understand the concepts needed to explore
  real-world applications of web development.

\item {\bf Review the Honor Code on the syllabus}. While you may discuss your
  design portfolio with others, copying source code or writing is a violation of
  Allegheny College's Honor Code.

\end{itemize}

\vspace*{-.15in}

\section*{Reading Assignment}

To ensure that you are best prepared to complete this project, please review all
of the chapters that we have covered up to and after the release date of this
assignment. As we cover new material during the remainder of the semester (e.g.,
advanced CSS and JavaScript programming), you are encouraged to review that
content as it will better enable you to complete a high-quality web design
portfolio.
%
If they have not done so already, students should read all of the relevant
``GitHub Guides'', available at \url{https://guides.github.com/}, that explain
how to use many of the features that GitHub provides. In particular, please make
sure that you have read guides such as ``Understanding the GitHub Flow'',
``Mastering Markdown'', and ``Documenting Your Projects on GitHub'' as each of
these articles will help you to understand how to use both GitHub and GitHub
Classroom.
%
Ultimately, students are responsible for, in consultation with the course
instructor, finding and understanding all of the technical resources that they
need to complete their web design portfolio.

\section*{Creating and Deploying a Web Site with Netlify and GitHub}

To start this project, you should learn about the ``JAMStack'' and then visit a
listing of web site templates that are available at
\url{https://templates.netlify.com/}. After carefully studying each of these
templates and the technologies that they will expect you to master, please pick
one and follow its instructions to create your new GitHub repository and deploy
the first version of your web site to Netlify.
%
Students should also take care to install the Netlify GitHub app so that you can
receive deploy summaries on your pull requests.
%
After the deploy finishes, check to make sure that your web site is live and
working as expected.
%
Now, you are ready to customize your web site by giving it a project-specific
uniform resource locator (URL). At this step you should also customize the name
of your GitHub repository so that it is the same as the full name of your web
site. For instance, the address of the instructor's web site is
\url{https://www.gregorykapfhammer.com/} and it is hosted in the GitHub
repository at \url{https://github.com/gkapfham/www.gregorykapfhammer.com}.
%
Remember, instead of using the default name for your web site's URL and its
GitHub repository, you should immediately pick suitable names that fit the theme
for your site!

After you have appropriately changed these names, please clone your repository
so that it is available on your workstation.
%
After the clone finishes, you can use \command{cd} to change into this new
directory. If you want to \step{go back} one directory from your current
location, then you can type the command \command{cd ..}. You may continue to use
the \command{cd} and \command{ls} commands to explore the files that you
automatically downloaded from GitHub.
%
After ensuring that you fully understand the organization of the source code
that emerges from your chosen template, please follow the ``GitHub Flow'' model
to create a branch of your repository, make that branch available on GitHub,
update at least one file in your branch, create a pull request, and then check
the preview build created by Netlify.
%
If you are not sure how to take these steps in the terminal window and/or
GitHub's web-based interface, please talk with the course instructor and review
the aforementioned resources.
%
You should continue to practice these important steps until you can perform them
with ease!
%
Here are some extra tips to help you to successfully complete your web design
portfolio:

% Please note that the course instructor expects students to implement and
% evaluate all of the source code needed to complete their proposed project. As
% such, there are no provided HTML, CSS, or JavaScript source code files for
% this assignment. With that said, you will need to edit, by the stated
% deadline, the three Markdown files in the \program{writing/} directory.
% Finally, you will need to submit, for instance, screenshots of your web site
% and reports on how your colleagues reacted to it. Here are some additional
% tips to assist you as you design, implement, and test the source code of the
% web site that you create for the final project:

\begin{itemize}
  \setlength{\itemsep}{0.05in}

  \item If you experience difficulties in getting your chosen template to work,
    please search its GitHub repository for a work-around that you can try.
    Otherwise, please pick an alternative template.

  \item Spend some time learning how your chosen template uses and organizes the
    Markdown, HTML, CSS, and JavaScript files. Take time to find and understand
    the purpose and behavior of the other types of files that are also in your
    site's repository (e.g., Sassy CSS files).

  \item Make sure that your HTML, CSS, and JavaScript source code is organized
    into directories.

  \item Remember that all of your source code must meet well-established
    programming standards.

  \item Unless you require a customized mobile-ready layout, consider using
    Bootstrap for this task or, alternatively, adopting the responsive design
    framework associated with your chosen template.

  \item If you download and use images from Flickr or Instagram, give
    credit and the license details.

  \item Use the textbook's description of web development projects to
    help you when brainstorming.

  \item Recall that there are no GatorGrader checks for this assignment since
    each project is different.

  \item In GatorGrader's absence, you should create and run correctness checks
    for your source code.

  \item Importantly, you should use ``linting'' tools to better ensure your
    source code's correctness.

\end{itemize}

The next step for this project is to identify one aspect of the template that
you want to customize. For instance, you might want to first focus on improving
the fonts, colors, content, or header, footer, or overall layout. Ultimately,
the final version of your web site should be fully customized and contain useful
content that completely explores your theme. At the start of the project,
though, you should focus on ensuring that you can make a change to your web
site's source code and then see it reflected in the deployed version of your
site.
%
While you are not required to do so, students who want to purchase their own
domain names can collaborate with the instructor to ensure that their
Netlify-hosted site is available under the domain that they purchased.

At this point you have started your web design portfolio. You should now ensure
that you schedule time every week to add content (e.g., new articles or
features) and source code (e.g., CSS files for a mobile-ready layout) to your
web design portfolio, ensuring that it demonstrates your mastery of the
technical skills in the field of web development and that you can submit it on
time.

\section*{Description of the Topics for the Design Portfolio}

Each student is invited to pick one of the following projects. Please note that
a student selecting the student-designed project must first discuss the idea
with the course instructor, during the upcoming practical and laboratory
sessions, and receive feedback and then final approval.
%
Please remember that you are fully responsible for ensuring the feasibility of
the project that you propose.

\begin{enumerate}

  \setlength{\itemsep}{0in}

  \item {\bf International Character Reference}: This topic invites you to
    investigate the Unicode standard for the consistent encoding and display of
    most of the world's languages. Even though Unicode is documented through
    several sites, such as \url{https://en.wikipedia.org/wiki/Unicode}, this
    standard is lengthy and cumbersome to search and understand. If you pick
    this project, you can learn more about Unicode by reading Section 3.4 and
    then develop a site that enables people to, for instance, search for and
    learn about specific characters. This web site should feature a responsive
    layout ensuring that both Unicode character examples and the textual content
    is visible on laptops, desktops, and mobile devices. Whenever possible, the
    web site should enable user interactivity through, for instance, the
    capability to display Unicode symbols in different colors or in boxes with
    different backgrounds.

  \item {\bf International Art Museum}: Develop an extension of the ``art
    store'' project described in Sections 3.7, 4.8, and 7.9. Your web site
    should feature examples of international art and then descriptions of both
    the artists who created the work and the art itself. Your site should
    feature a responsive layout displays both the art and the textual content on
    laptops, desktops, and mobile devices. Whenever possible, the site should
    support user interactivity through, for instance, an art discussion board or
    a background choice for the featured artwork.

  \item {\bf International Travel Site}: You should create an extension of the
    ``travel photographs'' project described in, for example, Section 7.9. This
    web site could feature photographs taken during international travel,
    further explaining the cultural context that is documented by the
    photograph. Additionally, the site can furnish tips for successful
    international travel or publish interviews with travellers. Your web site
    should feature a responsive layout ensuring that both the photographs and
    the textual content is visible on laptops, desktops, and mobile devices.
    Whenever possible, the site should enable user interactivity through, for
    instance, a travel discussion board or a zooming display for the travel
    photographs.

  \item {\bf International Country Site}: Selecting this project means that you
    will create an extension of the ``countries database'' site given in the
    extended example of Section 8.10. This site could feature the flag or some
    other iconic image for each country. Additionally, the site can furnish
    important facts about each of the chosen countries. Students who are
    interested in this project can investigate the Gapminder web site, at
    \url{https://www.gapminder.org}, for inspiration. This web site should
    feature a responsive layout ensuring that both the images and the textual
    content is visible on laptops, desktops, and mobile devices. Whenever
    possible, the web site should enable user interactivity through, for
    instance, a visualization of statistics about countries or a feature
    supporting the uploading of images and details about a new country.

  \item {\bf Intercultural Perspectives Survey}: You should create a
    significant extension of the ``country survey'' form given in Section 5.4.
    This web site could feature a survey that asks a person to share their views
    on one or more cultural issues. This site could also present some of the
    results from the individuals who have already taken the survey. This web
    site should feature a responsive layout ensuring that both the images and
    the textual content is visible on laptops, desktops, and mobile devices. A
    team that picks this project should follow the advice in Section 5.5 to
    ensure that their forms and tables are accessible to a wide range of people
    (e.g., individuals who face visual or mobility challenges). If possible, the
    site should enable user interactivity through, for instance, a visualization
    of results from the survey.

  \item {\bf Student-Designed Project}: Students will develop an idea for their
    own project that focuses on, whenever possible, both international or
    intercultural topics and the field of web development. After receiving the
    course instructor's approval for your idea, you will complete the project
    and report on it.
    %
    Students who select this project must ensure that it ultimately meets all of
    the requirements outlined in the following section (e.g., a mobile-ready
    layout).

\end{enumerate}

\section*{Requirements for the Web Design Portfolio}

To ensure that you have mastered the concepts introduced in this course, your
project's source code should adhere to the following requirements. These
requirements may be modified, at the discretion of the course instructor, only
if a student receives prior permission and documents this approval in the final
report and the source code. Without prior approval, all submitted projects
should contain source code in the Markdown, HTML, CSS, and JavaScript
programming languages.
%
While it is acceptable to reuse source code from previous laboratory and
practical assignments and your chosen template, the origin of all your derived
code must be clearly documented. Critically, the majority of your project's
source code may not be comprised of work that you found in online sources or
completed as part of a previous course assignment.
%
Additionally, your web site must feature a mobile-ready design that you created
with either bespoke CSS source code or through the use of a responsive web
design framework like Bootstrap. Finally, your site must feature images that you
appropriately stored inside of your GitHub repository.
%
Unless there is an extenuating reason that prevents you from doing so, your web
site must be publicly available in a GitHub repository and hosted on Netlify;
please see the instructor immediately if you cannot create a public site.

\section*{Summary of the Required Deliverables}

\noindent Students do not need to submit printed source code or technical
writing for any assignment in this course. Instead, this assignment invites you
to submit, using GitHub, the following deliverables.
%
Please note that you should maintain two distinct GitHub repositories: one for
the source code and assets of your web site and another, created by GitHub
Classroom for all of the other deliverables.
%
See a previous assignment for the steps to take for creating the second of these
repositories.

\vspace*{-0.05in}

\begin{enumerate}

  \setlength{\itemsep}{0in}

\item Completed, fully commented, and properly formatted versions of all the
  source code files. Please ensure that you source code adheres to all of the
  requirements mentioned in this assignment sheet (e.g., the use of all the
  required programming languages for web development).

\item A one-page written proposal, saved in the file
  \program{writing/proposal.md}, with an informative title, an abstract, a
  description of the main idea, an initial listing of the tasks that you must
  complete, and a plan that you will follow to complete the entire web design
  portfolio.

\item Three two-paragraph status updates, saved in the appropriate Markdown
  files, that explains what you have already implemented and the steps that you
  will take to finish your web site.

\item A detailed final project report, saved in the file
  \program{writing/report.md}, that documents, in a project-specific fashion,
  how you designed, implemented, tested, and evaluated your web site.

\item Stored in the \program{screenshots/} directory, a collection of five
  screenshots that showcase different aspects of your mobile-ready web site. You
  should produce screenshots that highlight how your site behaves at different
  viewport widths (e.g., desktop, laptop, tablet, and phone).

\item Stored in the \program{responses/} directory, a Markdown file called
  \program{feedback.md} that includes at least five direct quotations from
  non-course Allegheny College students who interact with your mobile-ready
  site. As previously described in the assignment sheet, you should run your web
  site on a laboratory computer and then ask your colleagues to share some
  feedback about it.

\end{enumerate}

\vspace*{-.1in}

\noindent You must complete all of the aforementioned deliverables by the
following due dates:

\vspace*{-.05in}

\begin{enumerate}

  \setlength{\itemsep}{0in}

  \item {\bf Project Assigned:} Friday, February 22, 2019

    After meeting with the course instructor and the members of your class, pick
    a topic for your final project. Remember, if you select the student-designed
    project, you must first have your project approved by the course instructor.
    Next, make sure that you create a public GitHub repository that contains
    source code and web resources that can be deployed by Netlify.

  \item {\bf Project Proposal and Plan:} Friday, March 1, 2019

    Write and submit a one-page proposal for your project. While you can use the
    project descriptions on the previous pages as a starting point, your
    proposal should have an informative title, an abstract, a description of the
    main idea, an initial listing of the tasks that you must finish, a plan for
    completing the work, and, if necessary, a statement of instructor approval.

  \item {\bf Status Update and Intermediate Project Demonstrations}: \newline
    Friday, March 15, 2019 and Friday, April 5, 2019 and Friday, April 19, 2019

    As you continue working on your project, please write and submit a
    two-paragraph status update through your GitHub repository created by GitHub
    Classroom. In addition, you should give a demonstration, during the
    practical session, that highlights the most important source code that you
    have finished so far.
    %
    The evidence of this completed demonstration should be a screenshot of a new
    feature on your web site, saved in your GitHub repository.

  \item {\bf Demonstration of Final Web Site}: Friday, April 26, 2019

    After receiving feedback on your web site from your peers who are not in
    this course, give a final demonstration of all the completed pages,
    highlighting its key features and any areas for improvement. Please note how
    your site's design has evolved since you started the project. Use feedback
    from the instructor to yield the final site that you submit on the due date.

  \item {\bf Due Date of the Web Design Portfolio}: Friday, May 4, 2019 by 11:59
    pm

    You should submit the final version of your project through both your site's
    GitHub repository and the repository created by GitHub Classroom. This
    submission should include all of the relevant source code and web site
    screenshots, the final version of each written reports, the textually
    archived feedback from the people who used your web site, and any additional
    required or supplementary materials that serve to demonstrate the success of
    your project.

\end{enumerate}

\vspace*{-.1in}

\section*{Evaluation of Your Web Design Portfolio}

Using a report that the instructor shares with you through the commit log in
GitHub, you will privately received a grade on this assignment and feedback on
your submitted deliverables. Your grade for the assignment will be a function of
the whether or not it was submitted in a timely fashion and if your repository
contains correct versions of all of the required Markdown and source code files.
Other factors will also influence your final grade on the assignment. Along with
checking for the existence of the required images and the screenshots of your
web site, the instructor will also evaluate the accuracy of both your technical
writing and the comments in your source code. Please see the instructor if you
have questions about the evaluation of this final project assignment.
%
Finally, in adherence to the Honor Code, students should only complete this
assignment in an independent fashion. Deliverables (e.g., HTML code or
Markdown-based technical writing) that are nearly identical to the work of
others (and not a part of your template) will be taken as evidence of violating
the \mbox{Honor Code}. Please see the instructor with any questions about these
policies.

\end{document}
