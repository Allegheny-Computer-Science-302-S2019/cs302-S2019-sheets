\documentclass[11pt]{article}

% NOTE: The "Edit" sections are changed for each assignment

% Edit these commands for each assignment

\newcommand{\assignmentduedate}{April 29}
\newcommand{\assignmentassignedate}{April 26}
\newcommand{\assignmentnumber}{Eight}

\newcommand{\labyear}{2019}
\newcommand{\labday}{Friday}
\newcommand{\labdueday}{Monday}
\newcommand{\labtime}{9:00 am}

\newcommand{\assigneddate}{Assigned: \labday, \assignmentassignedate, \labyear{} at \labtime{}}
\newcommand{\duedate}{Due: \labdueday, \assignmentduedate, \labyear{} at \labtime{}}

% Edit these commands to give the name to the main program

\newcommand{\mainprogram}{\lstinline{index.html}}
\newcommand{\mainprogramsource}{\lstinline{src/www/index.html}}

\newcommand{\secondprogram}{\lstinline{site.css}}
\newcommand{\secondprogramsource}{\lstinline{src/www/css/site.css}}

\newcommand{\screenshot}{\lstinline{javascript\_submission.png}}
\newcommand{\screenshotsource}{\lstinline{images/javascript\_submission.png}}

% Commands to describe key development tasks

% --> Running gatorgrader.sh
\newcommand{\gatorgraderstart}{\command{gradle grade}}
\newcommand{\gatorgradercheck}{\command{gradle grade}}

% Commands to describe key git tasks

% NOTE: Could be improved, problems due to nesting

\newcommand{\gitcommitfile}[1]{\command{git commit #1}}
\newcommand{\gitaddfile}[1]{\command{git add #1}}

\newcommand{\gitadd}{\command{git add}}
\newcommand{\gitcommit}{\command{git commit}}
\newcommand{\gitpush}{\command{git push}}
\newcommand{\gitpull}{\command{git pull}}

\newcommand{\gitcommitmainprogram}{\command{git commit src/www/index.html -m "Your
descriptive commit message"}}

% Use this when displaying a new command

\newcommand{\command}[1]{``\lstinline{#1}''}
\newcommand{\program}[1]{\lstinline{#1}}
\newcommand{\url}[1]{\lstinline{#1}}
\newcommand{\channel}[1]{\lstinline{#1}}
\newcommand{\option}[1]{``{#1}''}
\newcommand{\step}[1]{``{#1}''}

\usepackage{pifont}
\newcommand{\checkmark}{\ding{51}}
\newcommand{\naughtmark}{\ding{55}}

\usepackage{listings}
\lstset{
  basicstyle=\small\ttfamily,
  columns=flexible,
  breaklines=true
}

\usepackage{fancyhdr}

\usepackage[margin=1in]{geometry}
\usepackage{fancyhdr}

\pagestyle{fancy}

\fancyhf{}
\rhead{Computer Science 302}
\lhead{Practical Assignment \assignmentnumber{}}
\rfoot{Page \thepage}
\lfoot{\duedate}

\usepackage{titlesec}
\titlespacing\section{0pt}{6pt plus 4pt minus 2pt}{4pt plus 2pt minus 2pt}

\newcommand{\labtitle}[1]
{
  \begin{center}
    \begin{center}
      \bf
      CMPSC 302\\Web Development\\
      Spring 2019\\
      \medskip
    \end{center}
    \bf
    #1
  \end{center}
}

\begin{document}

\thispagestyle{empty}

\labtitle{Practical \assignmentnumber{} \\ \assigneddate{} \\ \duedate{}}

\section*{Objectives}

As a step towards learning more about how the JavaScript language supports the
implementation of interactive and dynamic web sites, you will create a web site
featuring JavaScript-based automated Google web font loading. Finally, you will
continue to practice running a web server and using automated tools, like
GatorGrader and \program{htmlhint}, to assess your completion of the assignment.

\section*{Suggestions for Success}

\begin{itemize}
  \setlength{\itemsep}{0pt}

\item {\bf Use the laboratory computers}. The computers in this laboratory feature specialized software for completing
  this course's laboratory and practical assignments. If it is necessary for you to work on a different machine, be sure
  to regularly transfer your work to a laboratory machine so that you can check its correctness. If you cannot use a
  laboratory computer and you need help with the configuration of your own laptop, then please carefully explain its
  setup to a teaching assistant or the course instructor when you are asking questions.

\item {\bf Follow each step carefully}. Slowly read each sentence in the assignment sheet, making sure that you
  precisely follow each instruction. Take notes about each step that you attempt, recording your questions and ideas and
  the challenges that you faced. If you are stuck, then please tell a teaching assistant or instructor what assignment
  step you recently completed.

\item {\bf Regularly ask and answer questions}. Please log into Slack at the start of a laboratory or practical session
  and then join the appropriate channel. If you have a question about one of the steps in an assignment, then you can
  post it to the designated channel. Or, you can ask a student sitting next to you or talk with a teaching assistant or
  the course instructor.

\item {\bf Store your files in GitHub}. Starting with this laboratory
  assignment, you will be responsible for storing all of your files (e.g., HTML
  or CSS code and Markdown-based writing) in a Git repository using GitHub
  Classroom. Please verify that you have saved your source code in your Git
  repository by using \command{git status} to ensure that everything is updated.
  You can see if your assignment submission meets the established correctness
  requirements by using the provided checking tools on your local computer and
  in checking the commits in GitHub.

\item {\bf Keep all of your files}. Don't delete your programs, output files, and written reports after you submit them
  through GitHub; you will need them again when you study for the quizzes and examinations and work on the other
  laboratory, practical, and final project assignments.

\item {\bf Back up your files regularly}. All of your files are regularly backed-up to the servers in the Department of
  Computer Science and, if you commit your files regularly, stored on GitHub. However, you may want to use a flash
  drive, Google Drive, or your favorite backup method to keep an extra copy of your files on reserve. In the event of
  any type of system failure, you are responsible for ensuring that you have access to a recent backup copy of all your
  files.

\item {\bf Explore teamwork and technologies}. While certain aspects of the laboratory assignments will be challenging
  for you, each part is designed to give you the opportunity to learn both fundamental concepts in the field of computer
  science and explore advanced technologies that are commonly employed at a wide variety of companies. To explore and
  develop new ideas, you should regularly communicate with your team and/or the teaching assistants and tutors.

\item {\bf Hone your technical writing skills}. Computer science assignments require to you write technical
  documentation and descriptions of your experiences when completing each task. Take extra care to ensure that your
  writing is interesting and both grammatically and technically correct, remembering that computer scientists must
  effectively communicate and collaborate with their team members and the tutors, teaching assistants, and course
  instructor.

\item {\bf Review the Honor Code on the syllabus}. While you may discuss your
  assignments with others, copying source code or writing is a violation of
  Allegheny College's Honor Code.

\end{itemize}

\section*{Reading Assignment}

If you have not done so already, please read all of the relevant ``GitHub
Guides'', available at \url{https://guides.github.com/}, that explain how to use
many of the features that GitHub provides. In particular, make sure that you
have read guides such as ``Mastering Markdown'' and ``Documenting Your Projects
on GitHub''; each of them will help you to understand how to use GitHub. To do
well on this assignment, you should also review Chapters 1 through 6 in the
course textbook, paying close attention to the web media and CSS content in
Sections 6.1 through 6.4 and Figure 6.13. Importantly, for this practical
assignment the students are responsible for finding and reading all of the
auxiliary materials that they need to successfully design the requested web
site. Please see the course instructor or a teaching assistant if you have
questions about these reading assignments.

\section*{Accessing the Practical Assignment on GitHub}

To access the laboratory assignment, you should go into the
\channel{\#announcements} channel in our Slack team and find the announcement
that provides a link for it. Copy this link and paste it into your web browser.
Now, you should accept the laboratory assignment and see that GitHub Classroom
created a new GitHub repository for you to access the assignment's starting
materials and to store the completed version of your assignment. Specifically,
to access your new GitHub repository for this assignment, please click the green
``Accept'' button and then click the link that is prefaced with the label ``Your
assignment has been created here''. If you accepted the assignment and correctly
followed these steps, you should have created a repository with a name like
``Allegheny-Computer-Science-302-Spring-2019/computer-science-302-spring-2019-practical-7-gkapfham''.
Unless you provide the instructor with documentation of the extenuating
circumstances that you are facing, not accepting the assignment means that you
automatically receive a failing grade for it.

Before you move to the next step of this assignment, please make sure that you
read all of the content on the web site for your new GitHub repository, paying
close attention to the technical details about the commands that you will type
and the content that must be evident in your web site. Now you are ready to
download the starting materials to your laboratory computer. Click the ``Clone
or download'' button and, after ensuring that you have selected ``Clone with
SSH'', please copy this command to your clipboard. To enter into your course
directory you should now type \command{cd cs302S2019}. Now, by typing
\command{git clone} in your terminal and then pasting in the string that you
copied from the GitHub site you will download all of the code for this
assignment. For instance, if the course instructor ran the \command{git clone}
command in the terminal, it would look like:

\begin{lstlisting}
  git clone git@github.com:Allegheny-Computer-Science-302-S2019/computer-science-302-spring-2019-practical-7-gkapfham.git
\end{lstlisting}

After this command finishes, you can use \command{cd} to change into the new
directory. If you want to \step{go back} one directory from your current
location, then you can type the command \command{cd ..}. Please continue to use
the \command{cd} and \command{ls} commands to explore the files that you
automatically downloaded from GitHub. What files and directories do you see? In
what ways will you have to extend them? Please note that this assignment
furnishes HTML and CSS files with minimal starting content. Spend some time
exploring these files, sharing your discoveries with the \mbox{course
instructor}.

\section*{Image Boxes with Backgrounds Created by CSS Gradients}

% <div class="lineargradientgreen">
%   <img src="img/hotair_balloon.jpg" alt="Hot Air Balloon"/>
% </div>

% .lineargradientgreen {
%   background-image: linear-gradient(green, white);
%   border: solid #777;
%   border-width: 2pt 2pt 2pt 2pt;
%   height: 500px;
%   width: 500px;
%   display:table-cell;
%   vertical-align:middle;
%   text-align:center
% }

This practical assignment invites you to learn how to create ``image boxes''
that both have backgrounds created by CSS gradients and contain a horizontally
and vertically centered image. These 500 pixel by 500 pixel square image boxes
should have a 2 pt border on each side. Additionally, the first five boxes
should have one of the backgrounds that correspond to the CSS gradients
specified in Figure 6.13. Moreover, every box should contain the
\program{img/hotair_balloon.jpg} in a fashion that is both vertically and
horizontally centered. Finally, the sixth image box's background should feature
a CSS gradient that you design to look different than the provided gradients.
Ultimately, your site will have a total of six stacked squares, each with a CSS
gradient for a background and an image. Here is an example of the source code
that you would write in the provided HTML file:

\begin{verbatim}
    <div class="lineargradientgreen">
      <img src="img/hotair_balloon.jpg" alt="Hot Air Balloon"/>
    </div>
\end{verbatim}

As you complete this practical assignment, you should regularly commit files to
your GitHub repository, using the ``Git Cheat Sheet'' and the steps that you
used in previous assignments and that are reviewed in the next section. Also,
note that your web server will require a dedicated terminal when it is running.
Students should also remember to check the correctness of their HTML by running
the command \command{htmlhint src/www/index.html} in their terminal window.

% If the web site displays incorrectly, please discuss the problems with a
% teaching assistant or the instructor.

Remember, you can run your web server by typing the command \command{serve
src/www 4250}. At this point, you can start your browser and go to the site
\url{http://localhost:4250/}. Please check to see if your web site looks
correct. If it is not, then continue to edit and check it until the files are
correct. You should also ensure that, when you run the web server, it produces
output suggesting that it is returning the CSS files and an image. If not, then
please make sure that you revise your HTML source code and/or check the
configuration of your web server. Finally, when you are studying your web site
you should verify the correctness of all its components (e.g., the backgrounds
of the boxes should individually feature the gradients in Figure 6.13 and then
your own bespoke gradient).

\section*{Checking the Correctness of Your Web Site}

Instead of requiring you to manually check that your deliverables adhere to the
industry-accepted standards for web site programming, the GatorGrader tool that
you will use in this laboratory assignment makes it easy for you to
automatically check if your submission meets the standards for correctness. For
instance, GatorGrader will check to ensure that \mainprogram{} has the required
title and header and the required \program{div} and \program{img} tags for the
six image boxes.

To get started with the use of GatorGrader, type the command \gatorgraderstart{}
in your terminal window. If you have mistakes in your assignment, then you will
need to review GatorGrader's output, find the mistake, and try to fix it. Once
your program is building correctly, fulfilling at least some of the assignment's
requirements, you should transfer your files to GitHub using the \gitcommit{}
and \gitpush{} commands. For example, if you want to signal that the
\mainprogramsource{} file has been changed and is ready for transfer to GitHub
you would first type \gitcommitmainprogram{} in your terminal, followed by
typing \gitpush{} to perform the transfer. Please see the instructor if you
cannot use GitHub.

After the course instructor enables \step{continuous integration} with a system
called Travis CI, when you use the \gitpush{} command to transfer your source
code to your GitHub repository, Travis CI will initialize a \step{build} of your
assignment, checking to see if it meets all of the requirements. If both your
source code and writing meet all of the established requirements, then you will
see a green \checkmark{} in the listing of commits in GitHub after awhile. If
your submission does not meet the requirements, a red \naughtmark{} will appear
instead. The instructor will reduce a student's grade for this assignment if the
red \naughtmark{} appears on the last commit in GitHub immediately before the
assignment's due date. Yet, if the green \checkmark{} appears on the last commit
in your GitHub repository, then you satisfied all of the main checks, thereby
allowing the course instructor to evaluate other aspects of your source code and
writing, as further described in the \step{Evaluation} section of this
assignment sheet. Unless you provide the instructor with documentation of the
extenuating circumstances that you are facing, no late work will be considered
towards your grade for this laboratory assignment.

\section*{Summary of the Required Deliverables}

\noindent To ensure that you can successfully create your own web sites that
feature web media and CSS, this practical provides minimal HTML and CSS source
code and invites you to add most of the required features. Students do not need
to submit printed source code or technical writing for any assignment in this
course. Instead, this assignment invites you to submit, using GitHub, the
following deliverables. Please see the instructor if you have questions about
this assignment's goals.

\vspace*{-.05in}

\begin{enumerate}

  \setlength{\itemsep}{0in}

\item A properly formatted and correct version of \mainprogramsource{} that both
  meets all of the established requirements and contains the correct HTML for
  the desired static web site.

\item A properly formatted and correct version of \secondprogramsource{} that
  both meets all of the established requirements and contains the correct CSS
  code to properly style image boxes.

\item Stored in \screenshot{}, a PNG image that give the screenshot of the final
  version of your web site. Note that you can use Ubuntu's ``print screen''
  feature to produce this image.

\end{enumerate}

\vspace*{-.1in}

\section*{Evaluation of Your Practical Assignment}

Using a report that the instructor shares with you through the commit log in
GitHub, you will privately received a grade on this assignment and feedback on
your submitted deliverables. Your grade for the assignment will be a function of
the whether or not it was submitted in a timely fashion and if your program
received a green \checkmark{} indicating that it met all of the requirements.
Other factors will also influence your final grade on the assignment. In
addition to studying the efficiency and effectiveness of your Markdown and/or
HTML source code, the instructor will also evaluate the accuracy of both your
writing and the constructs in your source code. But, if your submission receives
a red \naughtmark{}, then you will not receive the completion grade for this
practical assignment. Please see the course instructor if you have questions
about the evaluation of this practical assignment.

\section*{Adhering to the Honor Code}

In adherence to the Honor Code, students should complete this assignment on an
individual basis. While it is appropriate for students in this class to have
high-level conversations about the assignment, it is necessary to distinguish
carefully between the student who discusses the principles underlying a problem
with others and the student who produces assignments that are identical to, or
merely variations on, someone else's work. Deliverables (e.g., HTML or CSS
source code or technical writing) that are nearly identical to the work of
others will be taken as evidence of violating the \mbox{Honor Code}. Please see
the course instructor if you have questions about this policy.

\end{document}
